\documentclass[12pt,twoside,fleqn]{article}
\usepackage{code}
\usepackage{latexsym}
\usepackage{fullpage}
\newcommand{\sectionrule}[2]{\subsection*{{\bf#1}\hfill\fbox{$#2$}}}
\newcommand{\infrule}[2]{
  \begin{equation}
    \frac{\begin{array}{c}
              #1
          \end{array}}
         {\begin{array}{c}
              #2
          \end{array}}
  \end{equation}}
\newcommand{\labelledinfrule}[3]{
  \begin{equation}
    \label{#1}
    \frac{\begin{array}{c}
              #2
          \end{array}}
         {\begin{array}{c}
              #3
          \end{array}}
  \end{equation}}

\newcommand{\expression}{e}
\newcommand{\evalue}{e^v}
\newcommand{\evar}{x}
\newcommand{\econstant}{c}
\newcommand{\eselect}[2]{#1.#2}
\newcommand{\eapply}[2]{#1\,#2}
\newcommand{\efunction}[3]{\lambda{#1}{:}{#2}.{#3}}
\newcommand{\eif}[3]{\mathsf{if}\,#1\,\mathsf{then}\,#2\,\mathsf{else}\,#3}
\newcommand{\eseal}[2]{#1{:}#2}
\newcommand{\erecord}[1]{\{#1\}}

\newcommand{\cvar}{\alpha}
\newcommand{\constructor}{c}
\newcommand{\basetype}{b}
\newcommand{\cselect}[2]{#1.#2}
\newcommand{\carrow}[3]{({#1}{:}{#2}){\rightharpoonup}#3}
\newcommand{\carrowt}[3]{({#1}{:}{#2}){\to}#3}
\newcommand{\cnondeparrow}[2]{#1{\rightharpoonup}#2}
\newcommand{\cnondeparrowt}[2]{#1{\to}#2}
\newcommand{\cfunction}[4]{\lambda(#1{:}#2){:}#3.#4}
\newcommand{\capply}[2]{#1\,#2}
\newcommand{\crbnd}[4]{#1{\triangleright}#2{=}#3{:}#4}
\newcommand{\cndrbnd}[4]{#1{=}#3{:}#4}
\newcommand{\crecord}[1]{\{#1\}}
\newcommand{\ctype}{\tau}

\newcommand{\kind}{\kappa}
\newcommand{\ktype}{T}
\newcommand{\karrow}[2]{#1{\Rightarrow}#2}
\newcommand{\kpi}[3]{\Pi#1{:}#2.#3}
\newcommand{\ksigma}[3]{\Sigma#1{:}#2.#3}
\newcommand{\ksingleton}[2]{S(#1,#2)}
\newcommand{\krecord}[1]{\{{#1}\}}
\newcommand{\krbnd}[3]{#1{\triangleright}#2{:}#3}
\newcommand{\kndrbnd}[3]{#1{:}#3}

\newcommand{\mvar}{s}
\newcommand{\mstructure}[1]{[#1]}
\newcommand{\module}{M}
\newcommand{\mvalue}{M^v}
\newcommand{\mselect}[2]{#1.#2}
\newcommand{\mcast}[2]{#1{:>}#2}
\newcommand{\mapply}[2]{#1\,#2}
\newcommand{\mfunction}[3]{\lambda{#1}{:}{#2}.{#3}}
\newcommand{\mbindexpression}[2]{#1{=}#2}
\newcommand{\mbindconstructor}[2]{#1{=}#2}
\newcommand{\mbindmodule}[2]{#1{=}#2}

\newcommand{\signature}{\Sigma}
\newcommand{\ssig}{\mathsf{Sig}}
\newcommand{\sstructure}[1]{[#1]}
\newcommand{\sfunction}[3]{({#1}{:}{#2}){\rightharpoonup}#3}
\newcommand{\sfunctiont}[3]{({#1}{:}{#2}){\to}#3}
\newcommand{\sfunctionmaybet}[3]{({#1}{:}{#2}){\buildrel{\textstyle\to}\over\rightharpoonup} #3}
\newcommand{\snondepfunction}[2]{#1{\rightharpoonup}#2}
\newcommand{\snondepfunctiont}[2]{#1{\to}#2}
\newcommand{\sbindexpression}[2]{#1{:}#2}
\newcommand{\sbindconstructor}[2]{#1{:}#2}
\newcommand{\sbindmodule}[2]{#1{:}#2}

\newcommand{\field}[2]{#1{\triangleright}#2}
\newcommand{\mgenericbinding}{B}
\newcommand{\sgenericbinding}{D}
\newcommand{\mgenericbindings}[1]{B#1_1,\ldots,B#1_n}
\newcommand{\sgenericbindings}[1]{D#1_1,\ldots,D#1_n}
\newcommand{\fieldlabel}{l}

\newcommand{\context}{\Gamma}
\newcommand{\extendcontext}[2]{#1,#2}
\newcommand{\emptycontext}{\bullet}
\newcommand{\contextcontainsexpression}[3]
             {#1=\extendcontext{#1'}{\cbindexpression{#2}{#3}}{,}{#1''}}
\newcommand{\contextcontainsconstructor}[3]
             {#1=\extendcontext{#1'}{\cbindconstructor{#2}{#3}}{,}{#1''}}
\newcommand{\emptysequence}{\bullet}


\newcommand{\BV}[1]{\mbox{BV}(#1)}
\newcommand{\FV}[1]{\mbox{FV}(#1)}
\newcommand{\substitute}[2]{[{#1}{\mapsto}{#2}]}

\newcommand{\ok}[2]{#1\vdash #2\mbox{ ok}}
\newcommand{\validcontext}[1]{{#1}\vdash\mbox{ok}}
\newcommand{\validexpression}[3]{#1\vdash #2 : #3}
\newcommand{\validmodule}[3]{#1\vdash #2 : #3}
\newcommand{\validconstructor}[3]{#1\vdash #2 : #3}
\newcommand{\validtype}[2]{#1\vdash #2}
\newcommand{\validkind}[2]{#1\vdash #2}
\newcommand{\validsignature}[3]{\mbox{$#1\vdash #2 : #3$}}
\newcommand{\validmbinding}[3]{#1\vdash #2 : #3}
\newcommand{\validmbindings}[3]{#1\vdash #2 : #3}
\newcommand{\equivconstructor}[4]{#1\vdash #2 \equiv #3 : #4}
\newcommand{\equivkind}[3]{#1\vdash #2 \equiv #3}
\newcommand{\equivtype}[3]{#1\vdash #2 \equiv #3}
\newcommand{\equivsignature}[4]{#1\vdash #2 \equiv #3 : #4}
\newcommand{\equivsbinding}[3]{#1\vdash #2 \equiv #3}
\newcommand{\equivsbindings}[3]{#1\vdash #2 \equiv #3}
\newcommand{\subconstructor}[4]{\mbox{$#1\vdash #2 \preceq #3 : #4$}}
\newcommand{\subkind}[3]{\mbox{$#1\vdash #2 \preceq #3$}}
\newcommand{\subsignature}[4]{#1\vdash #2 \preceq #3 : #4}
\newcommand{\subsbinding}[3]{\mbox{$#1\vdash #2 \preceq #3$}}
\newcommand{\subsbindings}[3]{\mbox{$#1\vdash #2 \preceq #3$}}


\newcommand{\super}[2]{\ {}^{#1\in #2}}
\newcommand{\superlt}[2]{\ {}^{#1 < #2}}
\newcommand{\superle}[2]{\ {}^{#1 \le #2}}
\newcommand{\stepsto}{\leadsto}
\title{MIL Reference Guide (Working Draft)}
\author{Chris Stone}
\date{\today}

\begin{document}
\maketitle
\raggedbottom

\newcommand{\translate}[1]{\langle{#1}\rangle}
\newcommand{\describe}[7]{
  \subsubsection{#1}
  \begin{tabular}{p{0.8truein}p{5truein}}
  Description&#2\\
  Greek&$#3$\\
  Datatype&\textcd{#4}\\
  PP&\textcd{#5}\\
  Note&#6\\
  Rules&#7\\
\end{tabular}}
  

\newcommand{\irule}[2]
{  \[
    \frac{\begin{array}{c}
              #1
          \end{array}}
         {\begin{array}{c}
              #2
          \end{array}}
  \]}
\newcommand{\allarrowmil}[6]{\forall(#1).(#2;#3)\to^{#4,#5}#6}

\section{Kinds ($\kind$)} 

\subsection{Overview}

\subsubsection{Structural Rules}

\irule
  {\validkind{\context}{\kind}}
  {\equivkind{\context}{\kind}{\kind}}

\irule
  {\equivkind{\context}{\kind'}{\kind}}
  {\equivkind{\context}{\kind}{\kind'}}

\irule
  {\equivkind{\context}{\kind}{\kind'}\qquad
   \equivkind{\context}{\kind'}{\kind''}}
  {\equivkind{\context}{\kind}{\kind''}}


\irule
  {\equivkind{\context}{\kind}{\kind'}}
  {\subkind{\context}{\kind}{\kind}}

\irule
  {\subkind{\context}{\kind}{\kind'}\qquad
   \subkind{\context}{\kind'}{\kind''}}
  {\subkind{\context}{\kind}{\kind''}}


\subsection{Detailed Descriptions}

\describe{Type kind}
  {Kind of constructors which are the types of values}
  {\ktype}
  {Type\_k($\translate{\mathit{phase}}$)}
  {TYPE\_C \mbox{or} TYPE\_R}
  {}
  {\[\frac
      {\validcontext{\context}}
      {\validkind{\context}{\ktype}}\qquad
   \]}

\describe{Singleton kind}
  {Kind of constructors equivalent to a given constructor}
  {\ksingleton{\ktype}{\constructor}}
  {Singleton\_k($\translate{\mathit{phase}}$,$\translate{\ktype}$,$\translate{\constructor}$)}
  {SINGLE\_KC($\translate{\ktype}$,$\translate{\constructor}$) \mbox{\ or \ }
   SINGLE\_KR($\translate{\ktype}$,$\translate{\constructor}$)}
  {For concision, the compiler's code actually allows singletons at
   all kinds.  (These are definable in terms of singletons at kinds
   $\ktype$.)}
  {\[
     \frac
       {\validconstructor{\context}{\constructor}{\ktype}}
       {\validkind{\context}{\ksingleton{\ktype}{\constructor}}}
   \]
   \[
   \frac
    {\equivconstructor{\context}{\constructor}{\constructor'}{\ktype}}
    {\equivkind{\context}{\ksingleton{\ktype}{\constructor}}
       {\ksingleton{\ktype}{\constructor'}}}
   \]
   \[
   \frac
    {\validkind{\context}{\ksingleton{\ktype}{\constructor}}}
    {\subkind{\context}{\ksingleton{\ktype}{\constructor}}{\ktype}}
   \qquad
   \frac
    {\validconstructor{\context}{\constructor}
          {\ksingleton{\ktype}{\constructor'}}}
    {\equivconstructor{\context}{\constructor}{\constructor'}{\ktype}}
   \qquad
   \frac
    {\equivconstructor{\context}{\constructor}{\constructor'}{\ktype}}
    {\validconstructor{\context}{\constructor}
          {\ksingleton{\ktype}{\constructor'}}}
   \]}

\describe{Record Kind}
  {Kinds classifying records \textit{of} constructors (as
   distinguished from the type of a record value, which would have kind 
   $\ktype$).}
  {\krecord{\fieldlabel_i{\triangleright}\cvar_i{:}\kappa_i
            \super{i}{1..n}}}
  {Record\_k [(($\translate{\fieldlabel_i}$,$\translate{\cvar_i}$),$\translate{\kappa_i}$)$\super{i}{1..n}$]}
  {REC\_K\{$\translate{\fieldlabel_i}$>$\translate{\cvar_i}$:$\translate{\kappa_i}$
           $\super{i}{1..n}$\}} 
  {The kinds of the fields can have dependencies.  So for example,
   \[\krecord{\fieldlabel_1{\triangleright}{\alpha}{:}{\ktype},
              \fieldlabel_2{\triangleright}{\beta}{:}
                  {\ksingleton{\ktype}{\alpha}}}\]
   is the kind classifying a record containing two constructors
   of kind $\ktype$ where the second is constrained to be equivalent
   to the first.}
  {
    \irule
    {\validcontext{\context}\quad
       \validkind{\context,\sbindconstructor{\cvar_i}{\kind_i}\superlt{i}{n}}
                 {\kind_n}}
    {\validkind{\context}
       {\krecord{\krbnd{\fieldlabel_i}{\cvar_i}{\kind_i}\super{i}{1..n}}}}
    \irule
     {\forall j\in1..n:\quad
        \equivkind{\context,\sbindconstructor{\cvar_i}{\kind_i}
                  \superlt{i}{j}}
                 {\kind_j}{\kind'_j}}
     {\equivkind{\context}
        {\krecord{\krbnd{\fieldlabel_i}{\cvar_i}{\kind_i}\super{i}{1..n}}}
        {\krecord{\krbnd{\fieldlabel_i}{\cvar_i}{\kind'_i}\super{i}{1..n}}}}
    \irule
     {\forall j\in1..n:\quad
        \subkind{\context,\sbindconstructor{\cvar_i}{\kind_i}
                  \superlt{i}{j}}
                 {\kind_j}{\kind'_j}}
     {\subkind{\context}
        {\krecord{\krbnd{\fieldlabel_i}{\cvar_i}{\kind_i}\super{i}{1..n}}}
        {\krecord{\krbnd{\fieldlabel_i}{\cvar_i}{\kind'_i}\super{i}{1..n}}}}
   }

\newcommand{\kpimil}[3]{\Pi^{#1}(#2).#3}

\describe{Function Kind}
   {Kinds classifying functions mapping constructors \textit{to}
    constructors (as distinguished from the type of a term-level function,
    which would be classified by kind $\ktype$.)}
   {\kpimil{a}{\cvar_i{:}\kind_i\super{i}{1..n}}{\kind}\quad
    \mbox{\ where\ } a\in\{\mathcd{open},\mathcd{code},\mathcd{closure}\}}
   {Arrow\_k ($\translate{c}$, 
        [($\translate{\cvar_i}$,$\translate{\kind_i}$)$\super{i}{1..n}$],
        $\translate{\kind}$)}
   {Arrow\_k($\translate{c}$;$\translate{\cvar_i}$:$\translate{\kind_i}$$\super{i}{1..n}$;
             $\translate{\kind}$)}
   {Note that these are dependent function kinds.}
   {\irule
    {\validkind{\context}{\kind}\qquad
      \validkind{\context,\sbindconstructor{\cvar_i}{\kind_i}
                 \super{i}{1..n}}{\kind}}
     {\validkind{\context}{\kpimil{a}{\cvar_i{:}\kind_i\super{i}{1..n}}
                             {\kind}}}
    \irule
      {\forall j\in 1..n:\quad
         \equivkind{\context,\cvar_i{:}\kind_i\superlt{i}{j}}
                    {\kind_j}{\kind'_j}\\
         \equivkind{\context,\cvar_i{:}\kind_i\super{i}{1..n}}
                 {\kind}{\kind'}}
      {\equivkind{\context}
         {\kpimil{a}{\cvar_i{:}\kind_i\super{i}{1..n}}{\kind}}
         {\kpimil{a}{\cvar_i{:}\kind'_i\super{i}{1..n}}{\kind'}}}
    \irule
      {\forall j\in 1..n:\quad
         \subkind{\context,\cvar_i{:}\kind'_i\superlt{i}{j}}
                    {\kind'_j}{\kind_j}\\
         \subkind{\context,\cvar_i{:}\kind'_i\super{i}{1..n}}
                 {\kind}{\kind'}}
      {\subkind{\context}
         {\kpimil{a}{\cvar_i{:}\kind_i\super{i}{1..n}}{\kind}}
         {\kpimil{a}{\cvar_i{:}\kind'_i\super{i}{1..n}}{\kind'}}}
   }

\section{Constructors}

\subsection{Overview}

\subsubsection{Structural Rules}

\irule
  {\validconstructor{\context}{\constructor}{\kind}}
  {\equivconstructor{\context}{\constructor}{\constructor}{\kind}}

\irule
  {\equivconstructor{\context}{\constructor'}{\constructor}{\kind}}
  {\equivconstructor{\context}{\constructor}{\constructor'}{\kind}}

\irule
  {\equivconstructor{\context}{\constructor}{\constructor'}{\kind}\qquad
   \equivconstructor{\context}{\constructor'}{\constructor''}{\kind}}
  {\equivconstructor{\context}{\constructor}{\constructor''}{\kind}}

\subsection{Detailed Descriptions}

\describe{Integer}
  {Type of integers}
  {\mathsf{Int}_b\quad\mbox{\ where\ } b\in\{8,16,32\}}
  {Prim\_c(Int\_c $\translate{b}$, [])}
  {INT8, INT16, \textrm{or} INT32}
  {The datatype contains $\mathsf{Int}_{64}$, but this is not yet
   implemented.}
  {\irule
    {\validcontext{\context}\qquad b\in\{8,16,32\}}
     {\validconstructor{\context}{\mathsf{Int}_b}{\ktype}}}

\describe{Boxed Floating-Point}
  {Type of boxed floating-point values}
  {\mathsf{BoxFloat}_b\quad\mbox{\ where\ } b\in\{32,64\}}
  {Prim\_c(BoxFloat\_c $\translate{b}$, [])}
  {BOXFLOAT32, \mbox{or} BOXFLOAT64}
  {}
  {\irule
    {\validcontext{\context}}
     {\validconstructor{\context}{\mathsf{BoxFloat}_b}{\ktype}}}

\newcommand{\cexn}{\mathsf{Exn}}
\describe{Exception type}
  {Type of exception values}
  {\cexn}
  {Prim\_c(Exn\_c, [])}
  {EXN}
  {Corresponds to the Standard~ML type \textcd{exn}}
  {\irule
    {\validcontext{\context}}
    {\validconstructor{\context}{\cexn}{\ktype}}}

\newcommand{\cexntag}[1]{\mathsf{ExnTag}[#1]}
\describe{Exception Tag}
  {Type of exception tag values}
  {\cexntag{\constructor}}
  {Prim\_c(Exntag\_c, [$\translate{\constructor}$])}
  {EXNTAG($\translate{\constructor}$)}
  {These are the internal tags used to distinguish exception packets.
   Each tag comes with the type of values it can be used to tag.
   (Such tagged values then have type $\mathsf{Exn}$).  New tags
   are generated at run-time when new SML \textcd{exception} constructors
   are created.}
  {\[\frac
    {\validconstructor{\context}{\constructor}{\ktype}}
    {\validconstructor{\context}{\mathsf{ExnTag}[\constructor]}{\ktype}}
   \qquad
   \frac
    {\equivconstructor{\context}{\constructor}{\constructor'}{\ktype}}
    {\equivconstructor{\context}{\mathsf{ExnTag}[\constructor]}
         {\mathsf{ExnTag}[\constructor']}
         {\ktype}}
  \]}   

\describe{Array}
  {Type of array values}
  {\mathsf{Array}[\constructor]}
  {Prim\_c(Array\_c, [$\translate{\constructor}$])}
  {ARRAY($\translate{\constructor}$)}
  {An array of boxed floats is represented specially; there is also
   a special primitive to efficiently retrieve a floating point value
   from such an array.}
  {\[\frac
    {\validconstructor{\context}{\constructor}{\ktype}}
    {\validconstructor{\context}{\mathsf{Array}[\constructor]}{\ktype}}
   \qquad
   \frac
    {\equivconstructor{\context}{\constructor}{\constructor'}{\ktype}}
    {\equivconstructor{\context}{\mathsf{Array}[\constructor]}
         {\mathsf{Array}[\constructor']}
         {\ktype}}
  \]}   


\describe{Vector}
  {Type of vector values}
  {\mathsf{Vector}[\constructor]}
  {Prim\_c(Vector\_c, [$\translate{\constructor}$])}
  {VECTOR($\translate{\constructor}$)}
  {Vectors are immutable arrays.  Vectors of characters are strings.}\par
  {\[\frac
    {\validconstructor{\context}{\constructor}{\ktype}}
    {\validconstructor{\context}{\mathsf{Vector}[\constructor]}{\ktype}}
   \qquad
   \frac
    {\equivconstructor{\context}{\constructor}{\constructor'}{\ktype}}
    {\equivconstructor{\context}{\mathsf{Vector}[\constructor]}
         {\mathsf{Vector}[\constructor']}
         {\ktype}}
  \]}

\describe{References}
  {Type of reference values}
  {\mathsf{Ref}[\constructor]}
  {Prim\_c(Ref\_c, [$\translate{\constructor}$])}
  {REF($\translate{\constructor}$)}
  {}
  {\[\frac
    {\validconstructor{\context}{\constructor}{\ktype}}
    {\validconstructor{\context}{\mathsf{Ref}[\constructor]}{\ktype}}
   \qquad
   \frac
    {\equivconstructor{\context}{\constructor}{\constructor'}{\ktype}}
    {\equivconstructor{\context}{\mathsf{Ref}[\constructor']}
         {\mathsf{Ref}[\constructor]}
         {\ktype}}
  \]}

\newcommand{\csummil}[2]{\Sigma[#1;#2]}
\describe{Disjoint Sums}
  {Type of disjoint sum values}
  {\csummil{n}{\constructor_i\super{i}{1..m}}\quad
    \mbox{\ where\ }n,m\ge 0}
  {Prim\_c(Sum\_c\{tagcount=$\translate{n}$, totalcount=$\translate{n{+}m}$,
                   known=NONE\}, [$\translate{\constructor_i}\super{i}{1..m}$])}
  {SUM($\translate{n}$)($\translate{\constructor_i}\super{i}{1..m}$)}
  {The layout for elements of a sum type depends on the number of
  non-value-carrying components.  If we simply identified these with
  components of type $\mathsf{Unit}$, the possible layout of a type
  such as $\mathcd{'a option}$ could not be determined statically
  ($\mathcd{unit option = unit+unit}$, $\mathcd{int option =
  unit+int}$).  To avoid run-time type tests to determine layout, we
  statically fix a number $n$ of non-value-carrying components, which
  statically determines the layout.  This means that values of types
  such as $\mathcd{unit option}$ will be represented less compactly
  than is possible, but this isn't very important.}
  {\irule
    {n,m\ge 0\\
     \forall i\in1..m:\quad
       \validconstructor{\context}{\constructor_i}{\ktype}}
    {\validconstructor{\context}{\csummil{n}{\constructor_i\super{i}{1..m}}}
         {\ktype}}
   \irule
    {n,m\ge 0\\
     \forall i\in1..m:\quad
     \equivconstructor{\context}{\constructor_i}{\constructor'_i}{\ktype}}
    {\equivconstructor{\context}{\csummil{n}{\constructor_i\super{i}{1..m}}}
         {\csummil{n}{\constructor'_i\super{i}{1..m}}}
         {\ktype}}
  }

\newcommand{\cspsummil}[3]{\Sigma_{#1}[#2;#3]}
\describe{Exposed Disjoint Sums}
  {Type of disjoint sum values whose tag is statically known.}
  {\cspsummil{j}{n}{\constructor_i\super{i}{1..m}}\quad
    \mbox{\ where\ }1\le j \le n+m}
  {Prim\_c(Sum\_c\{tagcount=$\translate{n}$, totalcount=$\translate{n{+}m}$,
                   known=SOME~$\translate{j}$\}, 
    [$\translate{\constructor_i}\super{i}{1..m}$])}
  {SUM\_$\translate{j}$($\translate{n}$)($\translate{\constructor_i}\super{i}{1..m}$)}
  {}
  {\irule
    {j\in1..n{+}m\\
     \forall i\in1..m:\quad
       \validconstructor{\context}{\constructor_i}{\ktype}}
    {\validconstructor{\context}{\cspsummil{j}{n}
         {\constructor_i\super{i}{1..m}}}
         {\ktype}}
   \irule
    {j\in1..n{+}m\\
     \forall i\in1..m:\quad
     \equivconstructor{\context}{\constructor_i}{\constructor'_i}{\ktype}}
    {\equivconstructor{\context}
         {\cspsummil{j}{n}{\constructor_i\super{i}{1..m}}}
         {\cspsummil{j}{n}{\constructor'_i\super{i}{1..m}}}
         {\ktype}}
  }

\describe{Record}
  {Type of record values}
  {\{\fieldlabel_i{:}\constructor_i\super{i}{1..n}\}\quad
    \mbox{\ where\ }n\ge 0,\ \fieldlabel_i\mbox{\ distinct and sorted}}
  {Prim\_c(Record\_c [$\translate{\fieldlabel_i}\super{i}{1..n}$], 
            [$\translate{\constructor_i}\super{i}{1..n}$])}
  {RECORD[$\translate{\fieldlabel_i}\super{i}{1..n}$]($\translate{\constructor_i}\super{i}{1..n}$) 
     \textrm{or} UNIT \textrm{if $n=0$}}
  {}
  {\irule
    {\validcontext{\context}\qquad
     \forall i\in1..n:\ 
       \validconstructor{\context}{\constructor_i}{\ktype}}
    {\validconstructor{\context}
       {\crecord{{\fieldlabel_i}{:}{\constructor_i}\super{i}{1..n}}}
       {\ktype}}
    \irule
     {\validcontext{\context}\qquad
      \forall i\in1..n:\ 
        \equivconstructor{\context}{\constructor_i}{\constructor'_i}{\ktype}}
     {\equivconstructor{\context}
        {\crecord{{\fieldlabel_i}{:}{\constructor_i}\super{i}{1..n}}}
        {\crecord{{\fieldlabel_i}{:}{\constructor'_i}\super{i}{1..n}}}
        {\ktype}}}

\describe{Vararg}
  {Type constructor used in classifying \textcd{make\_vararg}
   and \textcd{make\_onearg}}
  {\mathsf{Vararg}^{a,e}[\constructor,\constructor']
     \mbox{\ where\ }  a\in\{\mathcd{open},\mathcd{code},\mathcd{closure}\},
     e\in\{\mathcd{partial},\mathcd{total}\}}
  {Prim\_c(Vararg\_c($\translate{a}$,$\translate{e}$),
    [$\translate{\constructor}$,$\translate{\constructor'}$])}
  {RECORD$\translate{a}\translate{e}
      (\translate{\constructor},\translate{\constructor'})$}
  {Vararg is not implemented in TILT (yet).  The pretty
   printer is screwed up for this case.}
  {\irule
    {
    \validconstructor{\context}{\constructor}{\ktype}\qquad
    \validconstructor{\context}{\constructor'}{\ktype}}
    {\validconstructor{\context}
       {\mathsf{Vararg}^{a,e}[\constructor,\constructor']}{\ktype}}
   \qquad
   \irule
    {
    \equivconstructor{\context}{\constructor_1}{\constructor'_1}{\ktype}\qquad
    \equivconstructor{\context}{\constructor_2}{\constructor'_2}{\ktype}}
    {\equivconstructor{\context}
       {\mathsf{Vararg}^{a,e}[\constructor_1,\constructor_2]}
       {\mathsf{Vararg}^{a,e}[\constructor'_1,\constructor'_2]}
         {\ktype}}
   \irule
    {
     \equivconstructor{\context}{\constructor}
        {\crecord{\fieldlabel_i{:}\constructor_i\super{i}{1..n}}}{\ktype}
         \qquad n\in0..6
    }
    {\equivconstructor{\context}
       {\mathsf{Vararg}^{a,e}[\constructor,\constructor']}
       {\allarrowmil{}
          {\constructor_i\super{i}{1..n}}
          {0}{a}{e}{\constructor}}
         {\ktype}}
  }

\describe{Recursion}
  {Recursive Types}
  {\mu(\cvar_i{=}\constructor_i\super{i}{1..n})\quad
     \mbox{\ where\ } n\ge 0}
  {Mu\_c(true, Seq[$(\cvar_i,\constructor_i)\super{i}{1..n}$])}
  {MU\_C($\translate{\cvar_i} = \translate{\constructor_i}\super{i}{1..n}$) \mbox{or} 
   MU\_C\_NR($\translate{\cvar_i} = \translate{\constructor_i}\super{i}{1..n}$)}
  {``Tuples'' of (mutually) recursively defined types; however the
   1-tuple case just returns the type.
   The boolean flag can be set to false in the datatype if the components
   are not really recursive.}
  {\[\frac{
       \validconstructor{\context,\cvar{:}\ktype}
            {\constructor}{\ktype}}
    {\validconstructor{\context}
            {\mu(\cvar{=}\constructor)}
            {\ktype}}
   \qquad
   \frac{n>1\qquad
       \forall j\in1..n:\
       \validconstructor{\context,\cvar_i{:}\ktype\super{i}{1..n}}
            {\constructor_j}{\ktype}}
    {\validconstructor{\context}
            {\mu(\cvar_i{=}\constructor_i\super{i}{1..n})}
            {\krecord{i{\triangleright}\cvar_i{:}\ktype\super{i}{1..n}}}}
   \]
   \[\frac{
       \equivconstructor{\context,\cvar{:}\ktype}
            {\constructor}{\constructor'}{\ktype}}
    {\equivconstructor{\context}
            {\mu(\cvar{=}\constructor)}
            {\mu(\cvar{=}\constructor')}
            {\ktype}}
   \]
   \[
   \frac{n>1\qquad
       \forall j\in1..n:\
       \equivconstructor{\context,\cvar_i{:}\ktype\super{i}{1..n}}
            {\constructor_j}{\constructor'_j}{\ktype}}
    {\equivconstructor{\context}
            {\mu(\cvar_i{=}\constructor_i\super{i}{1..n})}
            {\mu(\cvar_i{=}\constructor'_i\super{i}{1..n})}
            {\krecord{i{\triangleright}\cvar_i{:}\ktype\super{i}{1..n}}}}
   \]
  }

\describe{Function}
  {Type of a term-level monomorphic function}
  {\allarrowmil{}
          {\constructor_i\super{i}{1..m}}
          {k}{a}{e}{\constructor} \mbox{\ where\ }
       a\in\{\mathcd{open},\mathcd{code},\mathcd{closure}\} \mbox{\ and\ }
       e\in\{\mathcd{partial},\mathcd{total}\}}
  {AllArrow\_c($\translate{a}$,$\translate{e}$,%
               [],%
               [$\translate{\constructor_i}\super{i}{1..m}$], $\translate{k}$,%
               $\translate{\constructor}$)}
  {AllArrow($\translate{a}$; $\translate{e}$;
               ();
               $\translate{\constructor_i}\super{i}{1..m}$; 
               $\translate{k}$;
               $\translate{\constructor}$)}
  {The integer $k$ specifies the number of arguments to be passed (unboxed)
   in floating-point registers.  (There are only constructors representing
   monomorphic function types; there are, however, types for polymorphic
   functions.)}
  {\irule
    {m,k\ge 0\\
     \forall j\in1..m:\quad
       \validconstructor{\context}
          {\constructor_j}{\ktype}\\
     \validconstructor{\context}
          {\constructor}{\ktype}}
    {\validconstructor{\context}
       {\allarrowmil{}
          {\constructor_i\super{i}{1..m}}
          {k}{a}{e}{\constructor}}
       {\ktype}}
   \irule
    {m,k\ge 0\\
     \forall j\in1..m:\quad
       \equivconstructor{\context}
          {\constructor_j}{\constructor'_j}{\ktype}\\
     \equivconstructor{\context}
          {\constructor}{\constructor'}{\ktype}}
    {\equivconstructor{\context}
       {\allarrowmil{}
          {\constructor_i\super{i}{1..m}}
          {k}{a}{e}{\constructor}}
       {\allarrowmil{}
          {\constructor'_i\super{i}{1..m}}
          {k}{a}{e}{\constructor'}}
       {\ktype}}
  }

\describe{Type Variable}
  {Constructor variable}
  {\cvar}
  {Var\_c($\translate{\cvar}$)}
  {$\translate{\cvar}$}
  {}
  {\irule
    {\validcontext{\context}\qquad
     \context = \context',\cvar{:}\kind,\context''}
    {\validconstructor{\context}{\cvar}{\kind}}}

\newcommand{\cletmil}[3]{\mathsf{let}^{#1}\,#2\,\mathsf{in}\,#3\,\mathsf{end}}
\newcommand{\cbnd}{\mathit{cbnd}}
\describe{Let}
  {Constructor-level let-binding}
  {\cletmil{p}{\cbnd_i\super{i}{1..n}}{\constructor}\quad
     \mbox{\ where\ } p\in\{\mathcd{Parallel},\mathcd{Sequential}\}}
  {Let\_c ($\translate{p}$, [$\translate{\cbnd_i}\super{i}{1..n}$], $\translate{\constructor}$)}
  {LET $\translate{\cbnd_i}\super{i}{1..n}$ IN 
       $\translate{\constructor}$ END \mbox{or}
   LETP $\translate{\cbnd_i}\super{i}{1..n}$ IN $\translate{\constructor}$ END}
  {The choices for constructor bindings are:
   \[\begin{array}{rl}
       \cbnd ::= &\cvar = \constructor : \kind\\
         |&\cvar = \lambda^{\mathcd{Open}}(\cvar_i{:}\kind_i\super{i}{1..m})
                     {:}\kind.\constructor\\
         |&\cvar = \lambda^{\mathcd{Code}}(\cvar_i{:}\kind_i\super{i}{1..m})
                     {:}\kind.\constructor\\
     \end{array}
   \]
   which are represented in the datatype respectively as:
   \[
   \begin{array}{l}
      \mathcd{Con\_cb ($\translate{\cvar}$,$\translate{\kind}$,$\translate{\constructor}$)}\\
      \mathcd{Open\_cb ($\translate{\cvar}$,
                        [$(\translate{\cvar_i},\translate{\kind_i})\super{i}{1..m}$],
                        $\translate{\constructor}$,$\translate{\kind}$)}\\
      \mathcd{Code\_cb ($\translate{\cvar}$,
                        [$(\translate{\cvar_i},\translate{\kind_i})\super{i}{1..m}$],
                        $\translate{\constructor}$,$\translate{\kind}$)}\\
   \end{array}
   \]
  Notice that this serves as the intro form for constructor-level
  functions; all such functions are hence named.  The formal rules are 
  messy and omitted for now.}
  {}

\describe{Constructor Record}
 {Record of constructors (as distinguished from the type of a
  term-level record)}
 {\crecord{\fieldlabel_i{=}\constructor_i\super{i}{1..n}}}
 {Crecord\_c [$(\translate{\fieldlabel_i},\translate{\constructor_i})\super{i}{1..n}$]}
 {CREC\_C\{$\translate{\fieldlabel_i}$=$\translate{\constructor_i}\super{i}{1..n}$\}}
 {The order in which fields occur is significant in testing
  equivalence.}
 {\irule
  {\forall j\in1..n:\quad\cases{
     \cvar_j\not\in\BV{\context}\cr
     \validconstructor{\context}{\constructor_j}
         {[\cvar_i{\mapsto}\constructor_i\superlt{i}{j}]\kind_j}}}
  {\validconstructor{\context}
       {\crecord{{\fieldlabel_i}{=}{\constructor_i}
                 \super{i}{1..n}}}
       {\krecord{{\fieldlabel_i}{\triangleright}{\cvar_i}{:}{\kind_i}
                 \super{i}{1..n}}}}
 \irule
  {\forall j\in1..n:\quad\cases{
     \cvar_j\not\in\BV{\context}\cr
     \equivconstructor{\context}{\constructor_j}{\constructor'_i}
         {[\cvar_i{\mapsto}\constructor_i\superlt{i}{j}]\kind_j}}}
  {\equivconstructor{\context}
       {\crecord{{\fieldlabel_i}{=}{\constructor_i}
                 \super{i}{1..n}}}
       {\crecord{{\fieldlabel_i}{=}{\constructor'_i}
                 \super{i}{1..n}}}
       {\krecord{{\fieldlabel_i}{\triangleright}{\cvar_i}{:}{\kind_i}
                 \super{i}{1..n}}}}}

\describe{Projection}
  {Projection from a record of constructors}
  {\cselect{\constructor}{\fieldlabel}}
  {Proj\_c($\translate{\constructor}$,$\translate{\fieldlabel}$)}
  {PROJ\_C($\translate{\constructor}$,$\translate{\fieldlabel}$)}
  {}
  {\irule
    {\validconstructor{\context}{\constructor}
        {\krecord{\krbnd{\fieldlabel_i}{\cvar_i}{\kind_i}
                  \super{i}{1..n}}}}
    {\validconstructor{\context}{\cselect{\constructor}{\fieldlabel_j}}
      {[\cvar_i{\mapsto}{\cselect{\constructor}{\fieldlabel_i}}
        \superlt{i}{j}]
       \kind_j}}
   \irule
    {\equivconstructor{\context}{\constructor}{\constructor'}
        {\krecord{\krbnd{\fieldlabel_i}{\cvar_i}{\kind_i}
                  \super{i}{1..n}}}}
    {\equivconstructor{\context}
      {\cselect{\constructor}{\fieldlabel_j}}
      {\cselect{\constructor'}{\fieldlabel_j}}
      {[\cvar_i{\mapsto}{\cselect{\constructor}{\fieldlabel_i}}
        \superlt{i}{j}]
       \kind_j}}
   \irule
    {\validconstructor{\context}
          {\crecord{{\fieldlabel_i}{=}{\constructor_i}
                 \super{i}{1..n}}}
          {\krecord{\krbnd{\fieldlabel_i}{\cvar_i}{\kind_i}
                 \super{i}{1..n}}}
     \qquad j\in1..n}
    {\equivconstructor{\context}
       {\cselect
          {\crecord{{\fieldlabel_i}{=}{\constructor_i}
                 \super{i}{1..n}}}{\fieldlabel_j}}
       {[\cvar_i{\mapsto}{\constructor_i}\superlt{i}{j}]{\constructor_j}}
       {[\cvar_i{\mapsto}{\constructor_i}\superlt{i}{j}]{\kind_j}}}}
   
\describe{Closure}
  {Closure for a constructor-level function (as distinguished
   from the type of a term-level closure)}
  {\mathsf{Closure}[\constructor_1,\constructor_2]}
  {Closure\_c ($\translate{\constructor_1}$,$\translate{\constructor_2}$)}
  {CLOSURE\_C($\translate{\constructor_1}$,$\translate{\constructor_2}$)}
  {}
  {\irule
     {\validconstructor{\context}{\constructor}
          {\kpimil{\mathcd{code}}{\cvar_i{:}\kind_i\super{i}{1..n}}{\kind}}\\
      \validconstructor{\context}{\constructor_1}{\kind_1}}
     {\validconstructor{\context}
          {\mathsf{Closure}[\constructor,\constructor_1]}
         {\kpimil{\mathcd{closure}}
              {\cvar_i{:}\substitute{\cvar_1}{\constructor_1}\kind_i
                \super{i}{2..n}}{\substitute{\cvar_1}{\constructor_1}\kind}}}
   \irule
     {\equivconstructor{\context}{\constructor}{\constructor'}
         {\kpimil{\mathcd{closure}}{\cvar_i{:}\kind_i\super{i}{1..n}}{\kind}}\\
      \equivconstructor{\context}{\constructor_1}{\constructor'_1}{\kind_1}}
     {\equivconstructor{\context}
         {\mathsf{Closure}[\constructor,\constructor_1]}
         {\mathsf{Closure}[\constructor',\constructor'_1]}
         {\kpimil{\mathcd{closure}}
              {\cvar_i{:}\substitute{\cvar_1}{\constructor_1}\kind_i
                \super{i}{2..n}}{\substitute{\cvar_1}{\constructor_1}\kind}}}}

\newcommand{\capplymil}[2]{#1\,(#2)}
\describe{Application}
  {Application of a constructor-level function}
  {\capplymil{\constructor}{\constructor_i\super{i}{1..n}}}
  {App\_c ($\translate{\constructor}$, [$\translate{\constructor_i}\super{i}{1..n}$])}
  {APP\_C ($\translate{\constructor}$, [$\translate{\constructor_i}\super{i}{1..n}$])}
  {Also need to write beta and eta rules.}
  {\irule
    {\validconstructor{\context}{\constructor}
           {\kpimil{a}{\cvar_i{:}\kind_i\super{i}{1..n}}{\kind}}\\
      a\in\{\mathcd{open},\mathcd{code},\mathcd{closure}\}\\
     \forall j\in1..n:\quad
        \validconstructor{\context}{\constructor_j}
           {[\cvar_i{\mapsto}\constructor_i\superlt{i}{j}]\kind_i}}
    {\validconstructor{\context}
        {\capplymil{\constructor}{\constructor_i\super{i}{1..n}}}
        {[\cvar_i{\mapsto}\constructor_i\super{i}{1..n}]\kind}}
  \irule
    {\equivconstructor{\context}{\constructor}{\constructor'}
           {\kpimil{a}{\cvar_i{:}\kind_i\super{i}{1..n}}{\kind}}\\
      a\in\{\mathcd{open},\mathcd{code},\mathcd{closure}\}\\
     \forall j\in1..n:\quad
        \equivconstructor{\context}{\constructor_j}{\constructor'_j}
           {[\cvar_i{\mapsto}\constructor_i\superlt{i}{j}]\kind_i}}
    {\equivconstructor{\context}
        {\capplymil{\constructor}{\constructor_i\super{i}{1..n}}}
        {\capplymil{\constructor'}{\constructor'_i\super{i}{1..n}}}
        {[\cvar_i{\mapsto}\constructor_i\super{i}{1..n}]\kind}}
   }



\section{Types}

\subsection{Overview}

\subsubsection{Structural Rules}

\irule
  {\validtype{\context}{\kind}}
  {\equivtype{\context}{\kind}{\kind}}

\irule
  {\equivtype{\context}{\kind'}{\kind}}
  {\equivtype{\context}{\kind}{\kind'}}

\irule
  {\equivtype{\context}{\kind}{\kind'}\qquad
   \equivtype{\context}{\kind'}{\kind''}}
  {\equivtype{\context}{\kind}{\kind''}}

\subsection{Detailed Descriptions}

\describe{Constructors}
  {Every constructor of kind $\ktype$ corresponds to a type.}
  {\constructor}
  {\textrm{(see above)}}
  {\textrm{(see above)}}
  {}
  {\irule
    {\validconstructor{\context}{\constructor}{\ktype}}
    {\validtype{\context}{\constructor}}
   \irule
    {\equivconstructor{\context}{\constructor_1}{\constructor_2}{\ktype}}
    {\equivtype{\context}{\constructor_1}{\constructor_2}}}

\describe{Unboxed Floating-Point}
  {Type of unboxed floating-point values}
  {\mathsf{Float}_b\quad\mbox{\ where\ } b\in\{32,64\}}
  {Prim\_c(Float\_c $\translate{b}$, [])}
  {FLOAT32, \mbox{or} FLOAT64}
  {}
  {\irule
    {\validcontext{\context}}
     {\validtype{\context}{\mathsf{Float}_b}}}

\describe{Function}
  {Type of a term-level (possibly polymorphic) function}
  {\allarrowmil{\cvar_i{:}\kind_i\super{i}{1..n}}
          {\ctype_i\super{i}{1..m}}
          {k}{a}{e}{\ctype} \mbox{\ where\ }
       a\in\{\mathcd{open},\mathcd{code},\mathcd{closure},%
             \mathcd{extern}\} \mbox{\ and\ }
       e\in\{\mathcd{partial},\mathcd{total}\}}
  {AllArrow\_c($\translate{a}$,$\translate{e}$,%
               [$(\translate{\cvar_i},\translate{\kind_i})\super{i}{1..n}$],%
               [$\translate{\ctype_i}\super{i}{1..m}$], $\translate{k}$,%
               $\translate{\ctype}$)}
  {AllArrow($\translate{a}$; $\translate{e}$;
               $(\translate{\cvar_i}$ :: $\translate{\kind_i}\super{i}{1..n})$;
               $\translate{\ctype_i}\super{i}{1..m}$; 
               $\translate{k}$;
               $\translate{\ctype}$)}
  {The integer $k$ specifies the number of arguments to be passed (unboxed)
   in floating-point registers. }
  {\irule
    {m,n,k\ge 0\\
     \forall j\in1..n:\quad
       \validkind{\context,\cvar_i{:}\kind_i\superlt{i}{j}}{\kind_i}\\
     \forall j\in1..m:\quad
       \validconstructor{\context,\cvar_i{:}\kind_i\super{i}{1..n}}
          {\ctype_j}{\ktype}\\
     \validconstructor{\context,\cvar_i{:}\kind_i\super{i}{1..n}}
          {\ctype}{\ktype}}
    {\validtype{\context}
       {\allarrowmil{\cvar_i{:}\kind_i\super{i}{1..n}}
          {\ctype_i\super{i}{1..m}}
          {k}{a}{e}{\ctype}}}
   \irule
    {m,n,k\ge 0\\
     \forall j\in1..n:\quad
       \equivkind{\context,\cvar_i{:}\kind_i\superlt{i}{j}}
          {\kind_i}{\kind'_i}\\
     \forall j\in1..m:\quad
       \equivtype{\context,\cvar_i{:}\kind_i\super{i}{1..n}}
          {\ctype_j}{\ctype'_j}{\ktype}\\
     \equivtype{\context,\cvar_i{:}\kind_i\super{i}{1..n}}
          {\ctype}{\ctype'}{\ktype}}
    {\equivtype{\context}
       {\allarrowmil{\cvar_i{:}\kind_i\super{i}{1..n}}
          {\constructor_i\super{i}{1..m}}
          {k}{a}{e}{\constructor}}
       {\allarrowmil{\cvar_i{:}\kind'_i\super{i}{1..n}}
          {\constructor'_i\super{i}{1..m}}
          {k}{a}{e}{\constructor'}}}
  }


\section{Terms}

\subsection{Overview}

\subsection{Detailed Descriptions}

\describe{Variable}
  {Term-level variable}
  {\evar}
  {Var\_e $\translate{\evar}$}
  {$\translate{\evar}$}
  {}
  {\irule
    {\validcontext{\context}\qquad
     \context = \context',\evar{:}\ctype,\context''}
    {\validexpression{\context}{\evar}{\ctype}}}

\describe{Constant}
  {Term-level constants}
  {n_b, u_b, f_b,
   \mathsf{vector}[\constructor][\expression_i\super{i}{1..n}],
   \mathsf{array}[\constructor][\expression_i\super{i}{1..n}]}
  {Const\_e(int($\translate{b}$,$\translate{n}$)),\
   Const\_e(uint($\translate{b}$,$\translate{u}$)),\
   Const\_e(float($\translate{b}$,$\translate{f}$)),\
   Const\_e(vector($\translate{\constructor}$,[$\translate{\expression_i}\super{i}{1..n}$])),\ 
   Const\_e(array($\translate{\constructor}$,[$\translate{\expression_i}\super{i}{1..n}$]))}
  {$\translate{n}$, $\translate{u}$, $\translate{s}$, EmptyVectorValue 
   \mbox{or}\ VectorValue \mbox{or}\ "\ldots", ArrayValue}
  {The constructors annotating vectors and arrays are there for the
   length zero case, where they cannot be inferred.  In HIL there are
   different types for signed and unsigned integers; at the MIL level,
   the types are merged but there are signed and unsigned primitive
   operations.  }
  {\[
   \frac{\validcontext{\context}}
        {\validexpression{\context}{n_b}{\mathsf{Int}_b}}
   \qquad
   \frac{\validcontext{\context}}
        {\validexpression{\context}{u_b}{\mathsf{Int}_b}}
   \qquad
   \frac{\validcontext{\context}}
        {\validexpression{\context}{f_b}{\mathsf{Float}_b}}
   \]
   \[
   \frac{\validconstructor{\context}{\constructor}{\ktype}\qquad
         \forall i\in1..n:\quad
             \validexpression{\context}{\expression_i}{\constructor}}
        {\validexpression{\context}
            {\mathsf{vector}[\constructor][\expression_i\super{i}{1..n}]}
            {\mathsf{Vector}[\constructor]}}
   \]
   \[
   \frac{\validconstructor{\context}{\constructor}{\ktype}\qquad
         \forall i\in1..n:\quad
             \validexpression{\context}{\expression_i}{\constructor}}
        {\validexpression{\context}
            {\mathsf{array}[\constructor][\expression_i\super{i}{1..n}]}
            {\mathsf{Array}[\constructor]}}
   \]
  }

\describe{Record}
  {Term-level record}
  {\erecord{\fieldlabel_i{=}\expression_i\super{i}{1..n}}}
  {Prim\_e(NilPrimOp(record [$\translate{\fieldlabel_i}\super{i}{1..n}$]),%
            [$\translate{\constructor_i}\super{i}{1..n}$],%
            [$\translate{\expression_i}\super{i}{1..n}$])}
  {record[$\translate{\constructor_i}\super{i}{1..n}$]%
         ($\translate{\expression_i}\super{i}{1..n}$)}
  {The \textcd{elide\_prim} flag omits the constructors from
   pretty-printing output.}
 {\irule
  {\forall i\in1..n:\quad
     \validexpression{\context}{\expression_i}{\constructor_i}}
  {\validconstructor{\context}
       {\crecord{{\fieldlabel_i}{=}{\expression_i}
                 \super{i}{1..n}}}
       {\crecord{{\fieldlabel_i}{:}{\constructor_i}
                 \super{i}{1..n}}}}}

\describe{Selection}
  {Selection from a term-level record}
  {\eselect{\expression}{\fieldlabel}}
  {Prim\_e(NilPrimOp(select $\translate{\fieldlabel_j}$,%
                      [$\translate{\constructor}$],%
                      [$\translate{\expression}$]))}
  {select[$\translate{\fieldlabel_j}$]([$\translate{\constructor}$],%
                      [$\translate{\expression}$])}
  {The \textcd{elide\_prim} flag omits constructor from
   pretty-printing output.  The constructor $\constructor$ may be omitted,
   which helps avoid a blow-up in program size.}
 {\irule
  {\equivconstructor{\context}{\constructor}
       {\crecord{\fieldlabel_i{:}\constructor_i\super{i}{1..n}}}{\ktype}\qquad
     j\in1..n\qquad
   \validexpression{\context}{\expression}{\constructor}}
  {\validconstructor{\context}
       {\eselect{\expression}{\fieldlabel_j}}
       {\constructor_j}}}

\newcommand{\einj}[2]{\mathsf{inj}^{#1}\,#2}
\newcommand{\einjrec}[2]{\mathsf{inj\_rec}^{#1}(#2)}
\describe{Sum Tagging}
  {Injection into a sum type}
  {\einj{\constructor}{}\mathrm{\ and\ }
   \einj{\constructor}{\expression} \mathrm{\ and\ }
   \einjrec{\constructor}{\expression_i\super{i}{1..n}}}
  {Prim\_e(NilPrimOp(inject, [], [$\translate{\expression}$])) \textrm{\ and\ }
   Prim\_e(NilPrimOp(inject\_record, [], [$\translate{\expression_i}\super{i}{1..n}$]))}
  {inject[$\translate{\constructor}$]($\translate{\expression}$) 
   \textrm{\ and\ }
   inject\_rec[$\translate{\constructor}$]%
       ($\translate{\expression_i}\super{i}{1..n}$)}
  {}
  {\irule
    {\equivconstructor{\context}{\constructor}
       {\cspsummil{j}{k}{\constructor_i\super{i}{0..n-1}}}{\ktype}
         \qquad j\in 0..(k-1)}
    {\validexpression{\context}{\einj{\constructor}{}}
       {\csummil{k}{\constructor_i\super{i}{0..n-1}}}}
   \irule
    {\equivconstructor{\context}{\constructor}
       {\cspsummil{j}{k}{\constructor_i\super{i}{0..n-1}}}{\ktype}
         \qquad j\in k..(n+k-1)\\
     \validexpression{\context}{\expression}{\constructor_j}}
    {\validexpression{\context}{\einj{\constructor}{\expression}}
       {\csummil{k}{\constructor_i\super{i}{0..n-1}}}}
   \irule
    {\equivconstructor{\context}{\constructor}
       {\cspsummil{j}{k}{\constructor_i\super{i}{0..n-1}}}{\ktype}
          \qquad j\in k..(n+k-1)\\
     \equivconstructor{\context}{\constructor_{j-k}}
         {\crecord{\fieldlabel_i{:}\constructor'_i\super{i}{1..m}}}{\ktype}\\
     \forall i\in1..m:\quad
        \validexpression{\context}{\expression_i}{\constructor'_i}}
    {\validexpression{\context}{\einjrec{\constructor}
         {\expression_i\super{i}{1..m}}}
       {\csummil{k}{\constructor_i\super{i}{0..n-1}}}}}

\newcommand{\eproj}[2]{\mathsf{proj}^{#1}\,#2}
\newcommand{\eprojrec}[3]{\mathsf{proj\_rec}^{#1}_#2\,#3}
\describe{Sum Untagging}
  {Projection from an exposed sum type}
  {\eproj{\constructor}{\expression} \mbox{\ and\ }
   \eprojrec{\constructor}{j}{\expression}}
  {Prim\_e(NilPrimOp(project\_sum, [], [$\translate{\expression}$])) 
    \textrm{\ and\ }
   Prim\_e(NilPrimOp(project\_sum\_record $\translate{j}$, [], 
            [$\translate{\expression}$]))}
  {project\_sum[$\translate{\constructor}$]($\translate{\expression}$) 
     \textrm{\ and\ }
   project\_sum\_rec[$\translate{j}$][$\translate{\constructor}$]%
           ($\translate{\expression_i}\super{i}{1..n}$)}
  {}
  {\irule
    {\equivconstructor{\context}{\constructor}
       {\cspsummil{j}{k}{\constructor_i\super{i}{0..n-1}}}{\ktype}
         \qquad j\in k..(n+k-1)\\
     \validexpression{\context}{\expression}{\constructor}}
    {\validexpression{\context}{\eproj{\constructor}{\expression}}
         {\constructor_{j-k}}}
   \irule
    {\equivconstructor{\context}{\constructor}
       {\cspsummil{j}{k}{\constructor_i\super{i}{0..n-1}}}{\ktype}
          \qquad j\in k..(n+k-1)\\
     \equivconstructor{\context}{\constructor_{j-k}}
         {\crecord{\fieldlabel_i{:}\constructor'_i\super{i}{1..m}}}{\ktype}\\
     \validexpression{\context}{\expression}{\constructor}\qquad
      p\in1..m}
    {\validexpression{\context}{\eprojrec{\constructor}{p}
         {\expression}}
         {\constructor'_p}}}
  
\describe{Case analysis on Sums}     
 {Case construct for sums}
 {\mathsf{case}^{\ctype}\,\expression\,\mathsf{of}\,
    i\to\expression_i\super{i}{0..k-1},
    i\to(\evar)\expression_i\super{i}{k..k+n-1}}
 {Switch\_e(Sumsw\_e(\{sum\_type=$\translate{\constructor}$, 
                     arg=$\translate{\expression}$,
                     result\_type=$\translate{\ctype}$, 
                     bound=$\translate{\evar}$,
                     arms=[($\translate{i}$, $\translate{\expression_i}$)$\super{i}{0..k+n-1}$],
                     default=NONE\}))}
 {SWITCH\_SUM $\translate{\constructor}$ $\translate{\expression}$
            $\translate{i}$/\\()()(){:}$\translate{\constructor'}$=%
         $\translate{\expression_i}\super{i}{1..n}$
            $\translate{i}$/\\()($\translate{\evar}{:}\translate{\constructor_i}$)(){:}%
              $\translate{\constructor'}$=$\translate{\expression_i}$%
         $\super{i}{k+1..n+k-1}$
         NODEFAULT}
 {}
 {\irule
   {\validexpression{\context}{\expression}{\constructor}\\
    \equivconstructor{\context}{\constructor}
      {\csummil{k}{\constructor_i\super{i}{k+1..n+k-1}}}{\ktype}\\
    \forall i\in1..k:\quad 
       \validexpression{\context}{\expression_i}{\constructor'}\\
    \forall i\in k{+}1..n{+}k{-}1:\quad
       \constructor_i := \cspsummil{i}
           {k}{\constructor_i\super{i}{k+1..n+k-1}}\\
    \forall i\in k{+}1..n{+}k{-}1:\quad
       \validexpression{\context,\evar{:}\constructor_i}
            {\expression_i}{\constructor'}}
   {\validexpression{\context}
    {\mathsf{case}^{\constructor}\,\expression\,\mathsf{of}\,
      i\to\expression_i\super{i}{1..k},
      i\to(\evar)\expression_i\super{i}{k+1..n+k-1}}
    {\constructor'}}
 }
 

\describe{Boxing and Unboxing}
 {Boxing and unboxing of floating-point values}
 {\mathsf{Box}_b\,\expression\mbox{\ and\ }\mathsf{Unbox}_b\,\expression
   \mbox{\ where\ }  b\in\{32,64\}}
 {Prim\_e(NilPrimOp(box\_float $\translate{b}$,[],[$\translate{\expression}$]))
   \textrm{\ and\ }
  Prim\_e(NilPrimOp(unbox\_float $\translate{b}$,[],[$\translate{\expression}$]))}
 {box\_float\_$\translate{b}$[]($\translate{\expression}$)
   \textrm{\ and\ }
  unbox\_float\_$\translate{b}$[]($\translate{\expression}$)}
 {}
 {\irule
    {\validexpression{\context}{\expression}{\mathsf{Float}_b}
     \qquad b\in\{32,64\}}
    {\validexpression{\context}{\mathsf{Box}_b\,\expression}
      {\mathsf{BoxFloat}_b}}
  \irule
    {\validexpression{\context}{\expression}{\mathsf{BoxFloat}_b}
     \qquad b\in\{32,64\}}
    {\validexpression{\context}{\mathsf{Unbox}_b\,\expression}
      {\mathsf{Float}_b}}}

\newcommand{\eroll}[2]{\mathsf{roll}_{#1}\,#2}
\newcommand{\eunroll}[1]{\mathsf{unroll}\,#1}
\describe{Rolling and Unrolling}
 {Coercions to and from a recursive type}
 {\eroll{\constructor}{\expression}\mbox{\ and\ }
  \eunroll{\expression}}
 {Prim\_e(NilPrimOp(roll,[$\translate{\constructor}$],[$\translate{\expression}$])) \textrm{\ and\ }
  Prim\_e(NilPrimOp(unroll,[],[$\translate{\expression}$]))}
 {roll[$\translate{\constructor}$]($\translate{\expression}$)
   \textrm{\ and\ }
  unroll[]($\translate{\expression}$)}
 {As of 2/10, $\mathcd{unroll}$ still takes the recursive type
  as an argument, like $\mathcd{roll}$.  This is unnecessary and
  we're planning to drop it.}
 {For the $n=1$ case:
  \irule
   {\equivconstructor{\context}{\constructor}
       {\mu(\cvar{=}\constructor')}{\ktype}\qquad
    \validexpression{\context}{\expression}
         {\substitute{\cvar}{\constructor}\constructor'}}
   {\validexpression{\context}{\eroll{\constructor}{\expression}}
       {\constructor}}
  \irule
   {\validexpression{\context}{\expression}{\mu(\cvar{=}\constructor)}}
   {\validexpression{\context}{\eunroll{\expression}}
       {\substitute{\cvar}{\mu(\cvar{=}\constructor)}\constructor}}
 }

\newcommand{\enewtag}[1]{\mathsf{new\_tag}[#1]}
\describe{Exception Tags}
 {Creating a new exception tag}
 {\enewtag{\constructor}}
 {Prim\_e(NilPrimOp(make\_exntag,[$\translate{\constructor}$],[]))}
 {make\_exntag[$\translate{\constructor}$]()}
 {}
 {\irule
   {\validconstructor{\context}{\constructor}{\ktype}}
   {\validexpression{\context}{\enewtag{\constructor}}
        {\mathsf{ExnTag}[\constructor]}}}


\newcommand{\etag}[2]{\mathsf{tag}\,(#1,#2)}
\describe{Exceptions}
 {Creating an exception value}
 {\etag{\expression_1}{\expression_2}}
 {Prim\_e(NilPrimOp(inj\_exn "$\ldots$",[],[$\translate{\expression_1}$,
    $\translate{\expression_2}$]))}
 {inj\_exn["$\ldots$"]($\translate{\expression_1}$,$\translate{\expression_2}$)}
 {The string is used for debugging purposes.}
 {\irule
   {\validexpression{\context}{\expression_1}{\mathsf{ExnTag}[\constructor]}
      \qquad
    \validexpression{\context}{\expression_2}{\constructor}}
   {\validexpression{\context}{\etag{\expression_1}{\expression_2}}
        {\mathsf{Exn}}}}


\describe{Case analysis on Exceptions}
 {Case construct for exceptions}
 {\mathsf{exncase}_\ctype\,\expression\,\mathsf{of}\,
    \expression_i\to(\evar)\expression'_i\super{i}{1..n},
    \mathsf{default} \to \expression'}
 {Switch\_e(Exncase\_e(\{arg=$\translate{\expression}$,
                         result\_type=$\translate{\ctype}$,
                         bound=$\translate{\evar}$,
                         arms=[($\translate{\expression_i}$,$\translate{\expression'_i}$)%
                                 $\super{i}{1..n}$],
                         default=SOME $\translate{\expression'}$\})\})}
 {SWITCH\_EXN $\translate{\expression}$
            $\translate{\expression_i}$/\\()%
              ($\translate{\evar}{:}\translate{\constructor_i}$)(){:}%
              $\translate{\constructor'}$=$\translate{\expression_i}$%
         $\super{i}{1..n}$
         DEFAULT=$\translate{\expression'}$}
 {}
 {\irule
   {\validexpression{\context}{\expression}{\cexn}\\
    \forall i\in 1..n:\quad\cases{
       \validexpression{\context}{\expression_i}
           {\cexntag{\constructor'_i}}\cr
       \validexpression{\context,\evar{:}\constructor'_i}
            {\expression_i}{\constructor'}}}
   {\validexpression{\context}
    {(\mathsf{exncase}\,\expression\,\mathsf{of}\,
     \expression_i\to(\evar)\expression'_i\super{i}{1..n},
      \mathsf{default} \to \expression')}
    {\constructor'}}
 } 


\newcommand{\eapplymil}[4]{{#1}[#2](#3;#4)}
\describe{Application}
 {Application of a term-level function}
 {\eapplymil{\expression}{\constructor_i\super{i}{1..n}}
         {\expression_i\super{i}{1..m}}{\expression'_i\super{i}{1..p}}}
 {App\_e ($\translate{a}$, 
             $\translate{\expression}$, 
        [$\translate{\constructor_i}\super{i}{1..n}$],
        [$\translate{\expression_i}\super{i}{1..m}$],
        [$\translate{\expression'_i}\super{i}{1..p}$])}
 {App\_$\translate{a}$($\translate{\expression}$;$\translate{\constructor_i}\super{i}{1..n}$;%
       $\translate{\expression_i}\super{i}{1..m}$;$\translate{\expression'_i}\super{i}{1..p}$)}
 {}
 {\irule
   {\validexpression{\context}{\expression}
       {\allarrowmil{\cvar_i{:}\kind_i\super{i}{1..n}}
          {\ctype'_i\super{i}{1..m}}
          {p}{a}{e}{\constructor}}\\
    \forall j\in1..n:\quad
        \validconstructor{\context}{\constructor_j}
          {[\cvar_i{\mapsto}\constructor_i\superlt{i}{j}]\kind_j}\\
    \forall j\in1..m:\quad
        \validexpression{\context}{\expression_j}
          {[\cvar_i{\mapsto}\constructor_i\superlt{i}{j}]\ctype'_j}\\
    \forall j\in1..p:\quad
        \validexpression{\context}{\expression'_j}{\mathsf{Float}_{64}}}
   {\validexpression{\context}
      {\eapply{\expression}{\constructor_i\super{i}{1..n}}
         {\expression_i\super{i}{1..m}}{\expression'_i\super{i}{1..p}}}
      {[\cvar_i{\mapsto}\constructor_i\super{i}{1..n}]\kind}}
 }


\newcommand{\eletmil}[3]{\mathsf{let}^{#1}\,#2\,\mathsf{in}\,#3\,\mathsf{end}}
\newcommand{\ebnd}{\mathit{ebnd}}
\describe{Let}
  {Term-level let-binding}
  {\eletmil{p}{\ebnd_i\super{i}{1..n}}{\expression}\quad
     \mbox{\ where\ } p\in\{\mathcd{Parallel},\mathcd{Sequential}\}}
  {Let\_e ($\translate{p}$, [$\translate{\ebnd_i}\super{i}{1..n}$], 
           $\translate{\expression}$)}
  {LET $\translate{\ebnd_i}\super{i}{1..n}$ IN 
       $\translate{\expression}$ END \mbox{or}
   LETP $\translate{\ebnd_i}\super{i}{1..n}$ IN $\translate{\expression}$ END}
  {The choices for constructor bindings are:
   \[\begin{array}{rl}
       \cbnd ::= &\cvar = \constructor\\
         |&\evar = \expression : \ctype\\
         |&\{\evar_j = \Lambda(\cvar_{ji}{:}\kind_{ji}\super{i}{1..m_j})%
             \lambda(\evar'_{ji}{:}\ctype'_{ji}\super{i}{1..n_j};
                     \evar''_{ji}\super{i}{1..k_j})^{\mathcd{open},e_j}
                     {:}\ctype_j.\expression_j\super{j}{1..p}\}\\
         |&\{\evar_j = \Lambda(\cvar_{ji}{:}\kind_{ji}\super{i}{1..m_j})%
             \lambda(\evar'_{ji}{:}\ctype'_{ji}\super{i}{1..n_j};
                     \evar''_{ji}\super{i}{1..k_j})^{\mathcd{code},e_j}
                     {:}\ctype_j.\expression_j\super{j}{1..p}\}\\
         |&\{\evar_j = \mathsf{closure}%
                (\evar'_j,\constructor'_j,\expression_j:\constructor_j)
                \super{j}{1..p}\}\\
     \end{array}
   \]
   which are represented in the datatype respectively as:
   \[
   \begin{array}{l}
      \mathcd{Con\_b ($\translate{\cvar}$,$\translate{\kind}$,$\translate{\constructor}$)}\\
      \mathcd{Exp\_b ($\translate{\cvar}$,$\translate{\constructor}$,$\translate{\expression}$)}\\
      \mathcd{Fixopen\_b 
                 (Set[($\translate{\cvar_j}$, Function(Open, $\translate{e_j}$,
                        [$(\translate{\cvar_{ji}},\translate{\kind_i})\super{i}{1..m}$],}\\
      \mathcd{\qquad
                        [$(\translate{\evar'_{ji}},\translate{\ctype'{ji}})\super{i}{1..n}$],
                        [$\translate{\evar''_{ji}}\super{i}{1..k_j}$], 
                        $\translate{\expression_j}$,$\translate{\ctype_j}$))
                     $\super{j}{1..p}$])}\\
      \mathcd{Fixcode\_b 
                 (Set[($\translate{\cvar_j}$, Function(Code, $\translate{e_j}$,
                        [$(\translate{\cvar_{ji}},\translate{\kind_i})\super{i}{1..m}$],}\\
      \mathcd{\qquad
                        [$(\translate{\evar'_{ji}},\translate{\ctype'_{ji}})\super{i}{1..n}$],
                        [$\translate{\evar''_{ji}}\super{i}{1..k_j}$], 
                        $\translate{\expression_j}$,$\translate{\ctype_j}$))
                     $\super{j}{1..p}$])}\\
      \mathcd{Fixclosure\_b (true,
                 (Set[($\translate{\evar_j}$, 
                         \{code=$\translate{\evar'_j}$,
                           cenv=$\translate{\constructor'_j}$,}\\
      \mathcd{\qquad
                           venv=$\translate{\expression_j}$,
                           tipe=$\translate{\constructor_j}$\})
                        $\super{j}{1..p}$]))}
   \end{array}
   \]}
 {Notice that this serves as the intro form for term-level functions
  before and after closure conversion.  The formal rules are 
  messy and omitted for now.}
  {}

\newcommand{\eraise}[2]{\mathsf{raise}^{#1}\,#2}
\newcommand{\ehandle}[4]{\mathsf{handle}^{#1}\,#2\,\mathsf{with}\,#3\to#4}
\describe{Raise and Handle}
 {Raising and handling of exceptions}
 {\eraise{\ctype}{\expression}\mbox{\ and\ }
  \ehandle{\ctype}{\expression_1}{\evar}{\expression_2}}
 {Raise\_e ($\translate{\expression}$,$\translate{\ctype}$) \mbox{\ and\ }
  Handle\_e ($\translate{\expression_1}$, 
             $\translate{\evar}$,
             $\translate{\expression_2}$,$\translate{\ctype}$))}
 {RAISE($\translate{\expression}$,$\translate{\ctype}$) \mbox{\ and\ }
  HANDLE $\translate{\expression_1}$ : $\translate{\ctype}$ WITH
  $\translate{\evar}$~:~EXN~. $\translate{\expression_2}$}
 {To aid type reconstruction, raise is tagged with the type of the entire 
  raise expression.  Handle is tagged too, but there's no good reason for
  this}
 {\irule
    {\validtype{\context}{\ctype}\qquad
     \validexpression{\context}{\expression}{\mathsf{Exn}}}
    {\validexpression{\context}{\eraise{\constructor}{\expression}}
       {\ctype}}
  \irule
    {\validexpression{\context}{\expression_1}{\ctype}\qquad
     \validexpression{\context,\evar{:}\mathsf{Exn}}{\expression_2}
          {\ctype}}
    {\validexpression{\context}
          {\ehandle{\ctype}{\expression_1}{\evar}{\expression_2}}
          {\ctype}}}

\newcommand{\evararg}[5]{\mathsf{vararg}^{#1,#2}[#3,#4](#5)}
\newcommand{\eonearg}[5]{\mathsf{onearg}^{#1,#2}[#3,#4](#5)}

\describe{Argument flattening}
 {Converting between calling conventions}
 {\evararg{a}{e}{\constructor}{\constructor'}{\expression}\mbox{\ and\ }
  \eonearg{a}{e}{\constructor}{\constructor'}{\expression}, \mbox{\ where\ }
     {a\in\{\mathcd{open},\mathcd{code},\mathcd{closure}\}},
     e\in\{\mathcd{partial},\mathcd{total}\}}
 {Prim\_e(NilPrimOp(make\_vararg($\translate{a}$,$\translate{e}$),
              [$\translate{\constructor}$,$\translate{\constructor'}$],
              [$\translate{\expression}$])) \textrm{\ and\ }
  Prim\_e(NilPrimOp(make\_onearg($\translate{a}$,$\translate{e}$),
              [$\translate{\constructor}$,$\translate{\constructor'}$],
              [$\translate{\expression}$]))}
 {}
 {Currently unimplemented}
 {\irule
    {\validexpression{\context}{\expression}
          {\allarrowmil{}{\constructor}
             {0}{a}{e}{\constructor'}}}
    {\validexpression{\context}{\evararg{a}{e}{\constructor}{\constructor'}
           {\expression}}
           {\mathsf{Vararg}^{a,e}[\constructor,\constructor']}}
   \irule
    {\validexpression{\context}{\expression}
          {\mathsf{Vararg}^{a,e}[\constructor,\constructor']}}
    {\validexpression{\context}{\eonearg{a}{e}{\constructor}{\constructor'}
           {\expression}}
          {\allarrowmil{}{\constructor}{0}{a}{e}{\constructor'}}}}
           

\describe{typecase}
  {Term-level typecase}
  {\mathsf{typecase}_{\ctype}\,\constructor\,\mathsf{of}\,
    \mathsf{BoxFloat}_{64}\to\expression_1, \mathsf{default}\to\expression_2}
%  {typecase\_e \{result_type = $\translate{\ctype}$,
%                 arg=$\translate{\constructor}$,
%                arms=[(Prim\_c(Float\_c $\translate{64}$, []),
%                       [],$\translate{\constructor_1}$)],
%                default=$\translate{\constructor_2}$,
%                kind=$\kind$\}}
   {}
%  {TYPECASE\_C ($\translate{\constructor}$, $\translate{\kind}$,
%                FLOAT64 => FUN\_C()$\translate{\constructor_1}$,
%                DEFAULT: $\translate{\mathit{arm}}$)}
   {}
  {Not implemented yet.  The intended implementation is more
   general that that shown here.}
  {\irule
    {\validconstructor{\context}{\constructor}{\ktype}\qquad
     \validexpression{\context}{\expression_1}{\ctype}\qquad
     \validconstructor{\context}{\expression_2}{\ctype}}
    {\validexpression{\context}
      {\mathsf{typecase}_{\ctype}\,\constructor\,\mathsf{of}\,
         \mathsf{BoxFloat}_{64}\to\expression_1, 
         \mathsf{default}\to\expression_2}
     {\ctype}}}

\end{document}
